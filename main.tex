% Use the "normalphoto" option if you want a normal photo instead of cropped
% to a circle
% \documentclass[10pt,a4paper,normalphoto]{altacv}

\documentclass[10pt,a4paper,ragged2e,withhyper]{altacv}
%% AltaCV uses the fontawesome5 and packages.
%% See http://texdoc.net/pkg/fontawesome5 for full list of symbols.

% \geometry{left=1.25cm,right=1.25cm,top=1.5cm,bottom=1.5cm,columnsep=1.2cm}
\geometry{left=1.0cm,right=1.0cm,top=0.9cm,bottom=0.7cm,columnsep=1.0cm}

% Set columns of text in parallel
\usepackage{paracol}

% Change the font
\ifxetexorluatex
  % If using xelatex or lualatex:
  \setmainfont{Roboto Slab}
  \setsansfont{Lato}
  \renewcommand{\familydefault}{\sfdefault}
\else
  % If using pdflatex:
  \usepackage[rm]{roboto}
  \usepackage[defaultsans]{lato}
  % \usepackage{sourcesanspro}
  \renewcommand{\familydefault}{\sfdefault}
\fi

% Change colours
% \definecolor{SlateGrey}{HTML}{2E2E2E}
% \definecolor{LightGrey}{HTML}{666666}
% \definecolor{DarkPastelRed}{HTML}{450808}
% \definecolor{PastelRed}{HTML}{8F0D0D}
% \definecolor{DarkPastelGreen}{HTML}{56AE57}
% \definecolor{PastelGreen}{HTML}{77DD77}
% \definecolor{GoldenEarth}{HTML}{E7D192}

% \colorlet{name}{black}
% \colorlet{tagline}{PastelRed}
% \colorlet{heading}{blue}
% \colorlet{headingrule}{GoldenEarth}
% \colorlet{subheading}{PastelRed}
% \colorlet{accent}{PastelRed}
% \colorlet{emphasis}{SlateGrey}
% \colorlet{body}{LightGrey}

% Change fonts
\renewcommand{\namefont}{\Huge\rmfamily\bfseries}
\renewcommand{\personalinfofont}{\footnotesize}
\renewcommand{\cvsectionfont}{\LARGE\rmfamily\bfseries}
\renewcommand{\cvsubsectionfont}{\large\bfseries}

% Change bullets for itemize and rating marker for \cvskill
\renewcommand{\itemmarker}{{\small\textbullet}}
\renewcommand{\ratingmarker}{\faCircle}

% Can make fonts of itemize environments smaller
% \AtBeginEnvironment{itemize}{\small}

\begin{document}
\name{Anton Buyskikh}
\tagline{Senior Quantum Software Engineer}
\photoR{2.8cm}{figs/photo}

\personalinfo{%
    \email{anton.buyskikh@gmail.com}
    \phone{+44 7510 850009}
    \location{Cambridge, UK}
    \linkedin{anton-buyskikh}
    \github{anton-buyskikh}
    \printinfo{\faGraduationCap}{Google Scholar}
    [https://scholar.google.com/citations?user=G8zEw-0AAAAJ]
    % \printinfo{\faFlag}{Citizenships: Russia \& UK}[]
    
    % \orcid{0000-0003-4542-7086}
    % \homepage{www.homepage.com}
    % \twitter{@twitterhandle}
    
    % You can add your own arbitrary detail with
    % \printinfo{symbol}{detail}[optional hyperlink prefix]
    % \printinfo{\faFlag}{Russian-British Nationality}[]
    % Or you can declare your own field with
    % \NewInfoFiled{fieldname}{symbol}[optional hyperlink prefix] and use it:
    % \NewInfoField{gitlab}{\faGitlab}[https://gitlab.com/]
    % \gitlab{your_id}
    %
    % For services and platforms like Mastodon where there isn't a
    % straightforward relation between the user ID/nickname and the hyperlink,
    % you can use \printinfo directly e.g.
    % \printinfo{\faMastodon}{@username@instace}[https://instance.url/@username]
    % But if you absolutely want to create new dedicated info fields for
    % such platforms, then use \NewInfoField* with a star:
    % \NewInfoField*{mastodon}{\faMastodon}
    % then you can use \mastodon, with TWO arguments where the 2nd argument is
    % the full hyperlink.
    % \mastodon{@username@instance}{https://instance.url/@username}
}

\makecvheader

Senior software developer and data scientist with experience
delivering solutions for emerging computing technologies as well as adding
value to businesses is seeking for a new challenging opportunity.

% It's in my style to guide the solution from the early days of working with
% the customer understanding their needs, striking the balance between high
% quality and cost constraints.
% Senior Research Software Engineer with a wide experience of solution
% development for customers seeking for new opportunities of bringing value
% to your business.
% I currently develop the software stack for the near term quantum computing
% and have designed and delivered applications and libraries used for internal
% and external products.
% Being naturally good with communication and with attention to detail I like
% working on with multifunctional teams, bridging hardware and software teams.
% My previous experiences also include researching as a data scientist.

\columnratio{0.65}
\begin{paracol}{2}

% ==============================================================================
\cvsection{Relevant Experience}
% ==============================================================================

\cvevent{Senior Quantum Software Engineer}{Riverlane}
{Jul 2021 -- Present}{Cambridge, UK}
\cvevent{Quantum Software Engineer}{Riverlane}
{Sep 2019 -- Jul 2021}{Cambridge, UK}

% The products/projects I would like to talk about

% NQCC Benchkit
% 
% full cycle of product development, lead tech design, team work,
% product owner, collaboration
% 
% Product owner of a benchmarking product for an external customer:
% working with stakeholders on specs, organized and lead the technical team,
% external delivery, user training, and post-delivery maintenance.
% 
% Product owner of the product life cycle: from planning
% and writing the spec together with external stakeholders to the delivery,
% providing external training, and working on follow up feedback.

% Dfrun + Dfrun 2.0: Control/Decode integration
% 
% cross team work, low level integration,
% performance optimisation
% 
% Lead development of the internal hardware modelling tool that allowed us
% to estimate performance of the hardware integration ahead of RTL
% implementation.
% 
% Low level integration of the software stack responsible for quantum error
% correction (QEC) cycles.
% 
% Developed the compilation toolchain for lowering abstraction of physical
% qubits to the pulse level of the target control system.

% Calibration product and integration with Control
% 
% close work with external
% customers on shaping the product, development, demo, feedback,
% bringing into the production pipeline
% 
% Lead the technical side of the calibration product discovery with external
% customers.
% Developed multiple prototypes and convinced the stakeholders to integrate
% with the main stream of products.

% Deltaflow + NISQ.OS prototyping
% 
% design and research of the computation model for
% distributed control system, productisation

% General
% 
% Working together with the product and delivery teams to facilitate the
% best solution to the customer.
% CI/CD practices
% Processes

\begin{itemize}
    \item Early and long-term employee working on research, engineering, and
    integration for a quantum computing software startup

    \item Development of novel applications and libraries for
    distributed control system supporting calibration, benchmarking,
    and error correction for emerging fault-tolerant quantum computing
    technologies

    \item Tech lead and owner of multiple products and prototypes;
    full cycle of development from deriving users' requirements to
    delivery and maintenance

    \item Building robust, scalable codebases and engineering processes
    from the ground up

    \item Cross team work, integration and network latency optimisation

    \item Work with user requirements and research papers for deriving product
    specs accounting for time and resources constraints
    
    \item Collaborating closely with external clients to fully understand
    requirements of software/hardware design

    \item Prototyped and developed a minimal compilation toolchain for near
    term quantum computers 
    
    \item Line management and mentoring of junior members of staff and interns,
    leading by example, nurturing collaborative culture

    \item Public engagement, networking, recruitment, IP patenting,
    and shaping the direction and focus of the company during multiple growth
    phases

    \item Technologies used:
    {\bf Python, C++, SystemC, gRPC, FastAPI, Django, Docker, MySQL, Airflow,
    Celery, PostgreSQL, Rust, SystemVerilog, MPI, OpenMP, PHP, Xilinx FPGAs}
\end{itemize}

\divider

\cvevent{Research Associate}{University of Strathclyde}{Aug 2017 -- Sep 2019}
{Glasgow, UK}

\begin{itemize}
    \item Research and development of a software library for 1D tensor network
    algorithms to trace the dynamics of the quantum many-body systems

    \item Development of tensor network algorithms with 3rd party libraries
    for HPC, integration with our research needs

    \item Local HPC Linux cluster administration;
    training new staff and students

    \item Public engagement, outreach, networking, publishing our research
    in scientific papers and conference presentations
    
    \item Pushing frontiers of quantum simulations in external collaborations
    with experimental and theoretical groups

    \item Supervising and training undergraduate and PhD students, mentoring,
    shaping their professional goals

    \item Technologies used:
    {\bf Python, Matlab, C++, Boost, OpenMP, MPI, Slurm}
\end{itemize}

% \divider
\newpage

\cvevent{Data Scientist}{CitySprint}{Aug 2018 -- Sep 2018}{London, UK}

% Five weeks of intensive, project-based training turning exceptional analytical
% PhDs into Data Scientists.
% In addition a "mini-MBA" program, data science fellows work on a commercial
% data science problem with a company.
% Fellows are supported by business mentors, e.g. CTOs or Heads of
% Engineering/Analytics, technical mentors, experienced data scientists,
% and technologists from industry, and work in teams to develop high-impact,
% cutting-edge, sustainable, and scalable data science solutions.

\begin{itemize}
    \item Secondment in the analytics team for learning industry's best
    practices
    
    \item Developed a customer lifetime value model for investigation and
    mitigation of the customer churn

    \item Worked closely with Operations and Engineering teams on curating data,
    defining appropriate metrics of success, and integration of our solution
    into the company strategy

    \item Delivered our strategy proposal to the stakeholders; 
    potential annual saving estimated as £1M

    \item Technologies used: 
    {\bf Python, Machine Learning tools, Snowflake SQL, FastAPI}
\end{itemize}

\divider

\cvevent{Junior Researcher}{Ioffe Institute}{Sep 2009 -- Jul 2012}
{St. Petersburg, Russia}

\begin{itemize}
    \item Research and software development of model describing experimental
    setup for THz pulse generation in GaAs nanowires excited by 10 fs
    optical pulses

    \item Designed and built experiments with optical components and hardware
    drivers for characterisation of semiconductors, organic films and
    nanostructures
    
    \item Experienced with femtosecond lasers, time-domain spectroscopy, and
    lock-in amplification techniques

    \item Technologies used:
    {\bf Matlab, LabVIEW, COMSOL, Origin, Mathematica}
\end{itemize}

% ==============================================================================
\cvsection{Additional Experience}
% ==============================================================================

\cvevent{Tutor}{University of Strathclyde}{Aug 2017 -- Sep 2019}{Glasgow, UK}

Scientific programming with Python and Matlab;
focus on language syntax, design patterns, testing, and reusability

\divider

\cvevent{Treasurer \& Officer for Social Events}{SCOPE}{Oct 2015 -- Jan 2019}
{Glasgow, UK}
% \cvevent{Treasurer}{SCOPE}{2017/11 -- 2019/01}{Glasgow, UK}
% \cvevent{Officer for Social Events}{SCOPE}{2015/10 -- 2017/11}{Glasgow, UK}

Committee leadership;
public engagement, outreach, professional development and social event
organization

% Student Community for Optics, Physics and Engineering (SCOPE) is dedicated
% to enrich and enhance the postgraduate students experience by organising
% outreach and educational activities, jobs fairs, company and school visits,
% journal clubs, and social events.

\divider

\cvevent{Teacher Assistant}{University of Pittsburgh}{Aug 2012 -- Jun 2014}
{Pittsburgh, PA, USA}

Introduction To Physics, Classical Mechanics, Quantum Mechanics;
preparing and delivering tutorials, supervision, homework and exam grading

% \divider

% \cvevent{Photographer Assistant}{Old Abe's Old Time Portraits}
% {2009/06 -- 2009/08}{Wisconsin Dells, WI, USA}

% Work and Travel student exchange program.
% Duties involved guiding customers through the experience, assisting the main
% photographer with props and lightning, printing and framing, and general
% customer services.

\switchcolumn

% ==============================================================================
\cvsection{Education}
% ==============================================================================

\cvevent{PhD in Physics}{University of Strathclyde}{Glasgow, UK, 2017}{}
% \cvevent{PhD in Physics}{University of Strathclyde}{2014/08 -- 2017/08}
% {Glasgow, UK}
% Thesis: Dynamics of Quantum Many-Body Systems with Long-Range Interactions
% Supervisor: Prof. Andrew Daley

\divider

\cvevent{MSc in Physics}{University of Pittsburgh}{Pittsburgh, PA, USA, 2014}{}
% \cvevent{MSc in Physics}{University of Pittsburgh}{2012/08 -- 2014/08}
% {Pittsburgh, PA, USA}

\divider

\cvevent{BSc\,\&\,MSc in Theoretical Physics}
{St. Petersburg Polytechnic University}{St. Petersburg, Russia, 2012}{}
% \cvevent{BSc\,\&\,MSc in Condensed Matter Physics}
% {St. Petersburg Polytechnic University}
% {2006/09 -- 2012/06}{St. Petersburg, Russia}
% Thesis: THz radiation generation in AlGaAs semiconductor nanowires excited
% by femtosecond pulses.
% Supervisor: Dr. Valeriy N. Trukhin

% ==============================================================================
\cvsection{Top Technologies}
% ==============================================================================

% \cvevent{}{Programming Languages}{}{}
\cvevent{}{Languages}{}{}

\cvskill{Python}{5}
\cvskill{Bash}{4.5}
\cvskill{C++}{4}
\cvskill{SQL}{4}
\cvskill{Matlab}{4}
\cvskill{Rust}{3.5}
\cvskill{C\#}{3}

% \cvskill{Fortran}{4}
% \cvskill{Mathematica}{4}
% \cvskill{HTML}{3.5}
% \cvskill{SystemVerilog}{3}
% \cvskill{PHP}{3}
% \cvskill{JavaScript}{2.5}
% \cvskill{R}{2}
% \cvskill{Julia}{2}

\divider

\cvevent{}{Tools}{}{}

\cvskill{CMake}{5}
\cvskill{Docker}{5}
\cvskill{git/GitHub/GitLab}{5}
\cvskill{Django}{4.5}
\cvskill{Airflow}{4}
\cvskill{Azure}{4}
\cvskill{PostgreSQL/MySQL}{4}
\cvskill{SystemC}{4}
\cvskill{gRPC/MPI/OpenMP}{4}
\cvskill{TensorFlow}{3.5}

% \cvskill{Linux}{5}
% \cvskill{macOS}{5}
% \cvskill{Windows 10/11}{5}
% \cvskill{scikit-learn}{4}
% \cvskill{LLVM and GCC Toolchains}{4}
% \cvskill{Xilinx Ultra96 \& ZCU216}{3.5}
% \cvskill{MPI}{3.5}
% \cvskill{OpenMP}{3.5}
% \cvskill{Jenkins}{3}
% \cvskill{LLVM}{3}
% \cvskill{Unity}{3}
% \cvskill{Boost}{3}
% \cvskill{SoC Design}{2.5}

% ==============================================================================
\cvsection{Interests}
% ==============================================================================

\cvtag{Distributed Systems}
\cvtag{HPC}
\cvtag{CI/CD}
\cvtag{Optimisation}
\cvtag{Embedded Systems}
\cvtag{Statistical Analysis}
\cvtag{Error correction}
\cvtag{Quantum Computing}
\cvtag{Visualisation}
\cvtag{Machine Learning}
\cvtag{Compilers}\\
\cvtag{Data Science}

% \cvtag{Hardware Modelling}
% \cvtag{Analitics}

\newpage

% ==============================================================================
\cvsection{Soft Skills}
% ==============================================================================

\cvtag{Agile Software Development}\\
\cvtag{Jira Project Management}\\
\cvtag{Line Management}
\cvtag{Public Speaking}

% ==============================================================================
\cvsection{Training}
% ==============================================================================

% I try to undertake various training courses from time to time to diversify my
% skills, here's a list of the most relevant trainings/projects:

\begin{itemize}
    % \item Game development with Unity and C\# (2023/11)
    
    \item Essential People Skills for Line Managers (Oct 2023)
    % CIPD, Cambridge, UK

    \item Linux System Administration Essentials (Nov 2022)
    % Linux Foundation, Online
    
    \item ILM Level 3 Award in Leadership and Management Training (Aug 2021)
    % LearningTree, Online
    
    \item Comprehensive Verilog (May 2021)
    % Duolos, Online
    
    \item Embedded Systems Hardware and Software Design (May 2021)
    % Duolos, Online
    
    \item Data Science and Machine Learning Bootcamp (Aug 2018)
    % Pivigo, London, UK
    
    \item Neural Networks and Deep Learning (Jun 2017)
    % Coursera, Online
    
    \item Computational Condensed Matter Physics Bootcamp (Sep 2015)
    % ICTP, Triest, Italy
\end{itemize}

% ==============================================================================
\cvsection{About Me}
% ==============================================================================

\cvevent{}{Work Style}{}{}

\begin{itemize}
    \item Priority on delivery commitments

    \item Team building and knowledge exchange encouragement,
    attention to the team vision and individual skills

    \item Product oriented approach

    \item Inclined to automation, but understand exceptions for early
    explorations and quicker customer feedback
    
    \item Eye for detail; robust coding practices minimising human error

    \item Visualise complexity for collaboration
\end{itemize}

% I greatly enjoy working on complex products/features that require a team with
% multi-domain expertise.
% I enjoy taking a lead in breaking down the complexity, investigating the user
% requirements, planing the work with the team members, and working
% collaboratively throughout to the delivery.
% Moreover, I find the iterative approach the most efficient in the long run
% as it minimizes risks of mis-prioritisation.
% In my practice I find it essential to communicate with stakeholders from
% other functions, such as product, delivery, IT security, marketing, to have
% multiple viewpoints.

% During work on a project I would like to spend my time roughly like that:

% % Adapted from @Jake's answer from http://tex.stackexchange.com/a/82729/226
% % \wheelchart{outer radius}{inner radius}
% % {comma-separated list of value/text width/color/detail}
% \wheelchart{1.5cm}{0.5cm}{
%     45/10em/accent!10/{Sole development},
%     25/10em/accent!30/{Collaborative/Team work},
%     10/10em/accent!50/{Planning/Roadmapping},
%     10/10em/accent!70/{Meeting stakeholders},
%     10/10em/accent!90/{Lower priority work}
% }

\divider

\cvevent{}{Personal}{}{}

\begin{itemize}
    \item Family camping with rope climbing and canoeing
    
    \item Travelling and learning about new cultures and traditions

    \item Opportunistic woodworker

    \item Amateur gardener and veg grower

    \item Co-op video and board games
\end{itemize}

% My main priority in life is spending time with my family and passing my
% hobbies to my son.
% There are lots of plans for camping trips combined with water adventures and
% rope climbing, I wish my new job had remote opportunities...
% When the weather fails, I try to get a few woodworking
% projects done or play board or video games.
% Ah, yeah I can be quite a geek sometimes and have fun coding up maths puzzles
% such as "Project Euler".

% ==============================================================================
\end{paracol}
% ==============================================================================

% ==============================================================================
\end{document}
% ==============================================================================
