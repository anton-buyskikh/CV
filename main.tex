%% Use the "normalphoto" option if you want a normal photo instead of cropped to a circle
% \documentclass[10pt,a4paper,normalphoto]{altacv}

\documentclass[10pt,a4paper,ragged2e,withhyper]{altacv}
%% AltaCV uses the fontawesome5 and packages.
%% See http://texdoc.net/pkg/fontawesome5 for full list of symbols.

\geometry{left=1.25cm,right=1.25cm,top=1.5cm,bottom=1.5cm,columnsep=1.2cm}

% Set columns of text in parallel
\usepackage{paracol}

% Change the font, depending on whether you're using pdflatex or xelatex/lualatex
\ifxetexorluatex
  % If using xelatex or lualatex:
  \setmainfont{Roboto Slab}
  \setsansfont{Lato}
  \renewcommand{\familydefault}{\sfdefault}
\else
  % If using pdflatex:
  \usepackage[rm]{roboto}
  \usepackage[defaultsans]{lato}
  % \usepackage{sourcesanspro}
  \renewcommand{\familydefault}{\sfdefault}
\fi

% Change colours
\definecolor{SlateGrey}{HTML}{2E2E2E}
\definecolor{LightGrey}{HTML}{666666}
\definecolor{DarkPastelRed}{HTML}{450808}
\definecolor{PastelRed}{HTML}{8F0D0D}
\definecolor{DarkPastelGreen}{HTML}{56AE57}
\definecolor{PastelGreen}{HTML}{77DD77}
\definecolor{GoldenEarth}{HTML}{E7D192}

\colorlet{name}{black}
\colorlet{tagline}{PastelRed}
\colorlet{heading}{DarkPastelRed}
\colorlet{headingrule}{GoldenEarth}
\colorlet{subheading}{PastelRed}
\colorlet{accent}{PastelRed}
\colorlet{emphasis}{SlateGrey}
\colorlet{body}{LightGrey}

% Change fonts
\renewcommand{\namefont}{\Huge\rmfamily\bfseries}
\renewcommand{\personalinfofont}{\footnotesize}
\renewcommand{\cvsectionfont}{\LARGE\rmfamily\bfseries}
\renewcommand{\cvsubsectionfont}{\large\bfseries}

% Change bullets for itemize and rating marker for \cvskill
\renewcommand{\itemmarker}{{\small\textbullet}}
\renewcommand{\ratingmarker}{\faCircle}

% Can make fonts of itemize environments smaller
% \AtBeginEnvironment{itemize}{\small}

\begin{document}
\name{Anton Buyskikh}
\tagline{Senior Quantum Software Engineer}
\photoR{2.8cm}{figs/photo}

\personalinfo{%
    \email{anton.buyskikh@gmail.com}
    \phone{+44 7510 850009}
    \location{Cambridge, UK}
    \linkedin{anton-buyskikh}
    \github{anton-buyskikh}
    \printinfo{\faGraduationCap}{Google Scholar}[https://scholar.google.com/citations?user=G8zEw-0AAAAJ]
    \printinfo{\faFlag}{Citizenships: Russia \& UK}[]
    
    % \orcid{0000-0003-4542-7086}
    % \homepage{www.homepage.com}
    % \twitter{@twitterhandle}
    
    % You can add your own arbitrary detail with
    % \printinfo{symbol}{detail}[optional hyperlink prefix]
    % \printinfo{\faFlag}{Russian-British Nationality}[]
    % Or you can declare your own field with
    % \NewInfoFiled{fieldname}{symbol}[optional hyperlink prefix] and use it:
    % \NewInfoField{gitlab}{\faGitlab}[https://gitlab.com/]
    % \gitlab{your_id}
    %
    % For services and platforms like Mastodon where there isn't a
    % straightforward relation between the user ID/nickname and the hyperlink,
    % you can use \printinfo directly e.g.
    % \printinfo{\faMastodon}{@username@instace}[https://instance.url/@username]
    % But if you absolutely want to create new dedicated info fields for
    % such platforms, then use \NewInfoField* with a star:
    % \NewInfoField*{mastodon}{\faMastodon}
    % then you can use \mastodon, with TWO arguments where the 2nd argument is
    % the full hyperlink.
    % \mastodon{@username@instance}{https://instance.url/@username}
}

\makecvheader

% Research Software Engineer with a wide experience of solution development for customers
% seeking for new opportunities of bringing more value.
% It's in my style to guide the solution from the early days of working with the customer
% understanding their needs, striking the balance between high quality and cost constraints.
% I currently develop the software stack for the near term quantum computers and have designed
% and delivered applications and libraries used for internal and external products.
% Being naturally good with communication and with attention to detail I like working on
% with multifunctional teams, bridging hardware and software stack.
% My previous experiences also include researching as a data scientist.

% ==============================================================================
\columnratio{0.65}
\begin{paracol}{2}
% ==============================================================================

% ==============================================================================
\cvsection{Relevant Experience}
% ==============================================================================

\cvevent{Senior Quantum Software Engineer}{Riverlane}{2021/07 -- Present}{Cambridge, UK}
\cvevent{Quantum Software Engineer}{Riverlane}{2019/09 -- 2021/07}{Cambridge, UK}

\begin{itemize}

    \item
    Product owner of a benchmarking product for an external customer:
    working with stakeholders on specs, organized and lead the technical team,
    external delivery, user training, and post-delivery maintenance.
    {\bf [Python, FastAPI, Django, MySQL, PHP, Docker]}
    
    \item
    Lead the technical side of the calibration product discovery with external
    customers.
    Developed multiple prototypes and convinced the stakeholders to integrate
    with the main stream of products.
    {\bf [Python, Airflow, PostgreSQL, Docker, Celery, Xilinx ZCU216 FPGA]}
    
    \item
    Lead development of the internal hardware modelling tool that allowed us
    to estimate performance of the hardware integration ahead of RTL implementation.
    {\bf [C++, SystemC, Python]}

    \item
    Low level integration of the software stack responsible for quantum error
    correction (QEC) cycles.
    {\bf [C++, gRPC, Python]}
    
    \item
    Developed the compliation toolchain for lowering abstraction of physical
    qubits to the pulse level of the target control system.
    {\bf [Python]}
    
    \item
    Product owner of the product life cycle: from planning
    and writing the spec together with external stakeholders to the delivery,
    providing external training, and working on follow up feedback.
    
    \item
    Working together with the product and delivery teams to facilitate the
    best solution to the customer.
    
    \item
    Lead by example, follow best practices, and proactively offer support to our
    junior developers, mentoring and line managing.
    
    \item
    Line management and mentoring of junior members of staff and interns.
    
    \item
    Contributor to multiple internal software packages.
    {\bf [C++, Python, Rust, SystemVerilog]}

    % From Dan
    \item
    Early and long-term employee working on research, engineering and management for
    a quantum computing software startup

    \item
    Development of novel software algorithms and distributed control system tools
    for emerging quantum computing technologies
    
    \item
    Building robust, scalable codebases and engineering processes from the ground up
    
    \item
    Collaborating closely with academics and engineers to fully understand requirements of
    software/hardware co-design for maximum performance
    
    \item
    Public engagement, outreach, networking, recruitment, and shaping the direction and
    focus of the company during multiple growth phases
    
    \item
    Most used technologies:
    {\bf C++, Docker, Python, SystemC, SystemVerilog, MPI, OpenMP}

\end{itemize}

\divider

\cvevent{Data Scientist}{CitySprint}{2018/08 -- 2018/09}{London, UK}
 
% Five weeks of intensive, project-based training turning exceptional analytical
% PhDs into Data Scientists.
% In addition a "mini-MBA" program, data science fellows work on a commercial
% data science problem with a company.
% Fellows are supported by business mentors, e.g. CTOs or Heads of
% Engineering/Analytics, technical mentors, experienced data scientists,
% and technologists from industry, and work in teams to develop high-impact,
% cutting-edge, sustainable, and scalable data science solutions.

\begin{itemize}

    \item
    Expanded my skill by doing a secondment into a data science team for
    development of a customer retention strategy for based on analysis of the
    customer experience data.
    
    \item
    Developed a customer lifetime value (CLV) model for investigation and
    mitigation of the customer churn.

    \item
    Worked closely with BizDev, Marketing, and Tech teams on curating data,
    defining appropriate metrics of success and integration into the company
    strategy.

    \item
    Presented and handed over our strategy proposal to the stakeholders; 
    potential annual saving could were estimated as £1M.

    \item
    Technologies used: Python, scikit-learn and other ML libraries,
    Snowflake SQL, FastAPI.

\end{itemize}

\divider

\cvevent{Research Associate}{University of Strathclyde}{2017/08 -- 2019/09}{Glasgow, UK}

\begin{itemize}

    \item
    Research and development of a library of one dimensional tensor network
    algorithms used by our team to trace the dynamics of the quantum many-body
    systems.

    \item
    Development of tensor network algorithms with 3rd party libraries for HPC,
    integration with our research needs.

    \item
    Local HPC Linux cluster administration. Training of new staff and student.

    \item
    Presentation of our research work as scientific paper and at conferences.
    
    \item
    Pushing frontiers of quantum simulations by numerous external collaborations
    with experimental groups and theoretical predictions with empirical data.

    \item
    Supervising and training undergraduate and PhD students, mentoring, shaping,
    their professional goals.

    \item
    Technologies used: {\bf Matlab, Python, C++, Boost, Slurm}.

\end{itemize}

\divider

\cvevent{Junior Research Assistant}{Ioffe Institute}{2009/09 -- 2012/07}{St. Petersburg, Russia}

\begin{itemize}

    \item
    Modelling and numerical simulation of the THz pulse generation in i-GaAs
    and LT-GaAs nanowire excited by femtosecond (10-100 fs) optical pulses.

    \item
    Designed and built experiments with hardware drivers for
    characterisation of semiconductors, organic films and nanostructures.
    
    \item
    Experienced with femtosecond lasers: MaiTai of Newport Corporation (<100 fs)
    and Synergy of Femtolasers Produktions GmbH (<10 fs).
    Used THz time-domain spectroscopy and lock-in amplification techniques.

    \item
    Technologies used: {\bf Matlab, COMSOL, Mathematica, LabVIEW, Origin}.

\end{itemize}

% ==============================================================================
\cvsection{Additional Experience}
% ==============================================================================

\cvevent{Tutor}{University of Strathclyde}{2018/09 -- 2019/09}{Glasgow, UK}

Topics: Scientific programming on Python and Matlab for undergraduate students.

% \divider

% \cvevent{Treasurer}{SCOPE}{2017/11 -- 2019/01}{Glasgow, UK}

% \divider

% \cvevent{Officer for Social Events}{SCOPE}{2015/10 -- 2017/11}{Glasgow, UK}

\divider

\cvevent{Teacher Assistant}{University of Pittsburgh}{2012/08 -- 2014/06}{Pittsburgh, PA, USA}

Classes: Introduction To Physics, Classical Mechanics, Quantum Mechanics

% \divider

% \cvevent{Photographer Assistant}{Old Abe's Old Time Portraits}{2009/06 -- 2009/08}{Wisconsin Dells, WI, USA}

% Work and Travel student exchange program.
% Duties involved guiding customers through the experience, assisting the main photographer with
% props and lightning, printing and framing, and general customer services.

% ==============================================================================
\cvsection{Volunteering}
% ==============================================================================

\cvevent{Treasurer}{SCOPE}{2017/11 -- 2019/01}{Glasgow, UK}
\cvevent{Officer for Social Events}{SCOPE}{2015/10 -- 2017/11}{Glasgow, UK}

% Student Community for Optics, Physics and Engineering (SCOPE) is dedicated
% to enrich and enhance the postgraduate students experience by organising outreach and
% educational activities, jobs fairs, company and school visits, journal clubs,
% and social events.

% ==============================================================================
\switchcolumn
% ==============================================================================

% ==============================================================================
\cvsection{Education}
% ==============================================================================

\cvevent{PhD in Physics}{University of Strathclyde}{Glasgow, UK, 2017}{}
% \cvevent{PhD in Physics}{University of Strathclyde}{2014/08 -- 2017/08}{Glasgow, UK}
% Thesis: Dynamics of Quantum Many-Body Systems with Long-Range Interactions
% Supervisor: Prof. Andrew Daley

\divider

\cvevent{MSc in Physics}{University of Pittsburgh}{Pittsburgh, PA, USA, 2014}{}
% \cvevent{MSc in Physics}{University of Pittsburgh}{2012/08 -- 2014/08}{Pittsburgh, PA, USA}

\divider

\cvevent{BSc\,\&\,MSc in Theoretical Physics}{St. Petersburg Polytechnic University}{St. Petersburg, Russia, 2012}{}
% \cvevent{BSc\,\&\,MSc in Condensed Matter Physics}{St. Petersburg Polytechnic University}{2006/09 -- 2012/06}{St. Petersburg, Russia}
% Thesis: THz radiation generation in AlGaAs semiconductor nanowires excited by femtosecond pulses.
% Supervisor: Dr. Valeriy N. Trukhin

% ==============================================================================
\cvsection{Technologies}
% ==============================================================================

\cvevent{}{Programming Languages}{}{}

\cvtag{Python}
\cvtag{C++}
\cvtag{Bash}
\cvtag{SQL}
\cvtag{Rust}
\cvtag{Matlab}
\cvtag{C\#}
\cvtag{SystemVerilog}
\cvtag{Fortran}
\cvtag{Mathematica}
\cvtag{R}
\cvtag{Julia}
\cvtag{HTML}
\cvtag{JavaScript}
\cvtag{PHP}

\divider

\cvevent{}{Tools}{}{}

\cvtag{Docker}
\cvtag{Airflow}
\cvtag{LLVM}
\cvtag{SystemC}
\cvtag{PostgreSQL}
\cvtag{Azure}
\cvtag{git}
\cvtag{gRPC}
\cvtag{MPI}
\cvtag{OpenMP}
\cvtag{Unity}
\cvtag{Django}
\cvtag{GitHub Actions}
\cvtag{TensorFlow}
\cvtag{Boost}
\cvtag{scikit-learn}
\cvtag{Hardware Modelling}
\cvtag{SoC Design}
\cvtag{LLVM and GCC Toolchains}
\cvtag{Linux}
\cvtag{macOS}
\cvtag{Windows 10/11}
\cvtag{CI/CD}
\cvtag{Xilinx Ultra96 \& ZCU216}

% ==============================================================================
\cvsection{Soft Skills}
% ==============================================================================

\cvtag{Agile Software Development}

\cvtag{Jira Project Management}

\cvtag{Line Management}
\cvtag{Public Speaking}

% ==============================================================================
\cvsection{Interests}
% ==============================================================================

\cvtag{Distributed Systems}
\cvtag{HPC}\\
\cvtag{Embedded Systems}
\cvtag{Statistical Analysis}
\cvtag{Machine Learning}
\cvtag{Computer Vision}
\cvtag{Optimisation}
\cvtag{Compilation Toolchains}

% ==============================================================================
\cvsection{Trainings}
% ==============================================================================

% I try to undertake various training courses from time to time to diversify my skills,
% here's a list of the most relevant trainings/projects:

\begin{itemize}
    
    % \item
    % 2023/11 --- Game development with Unity and C\# --- Online
    
    \item
    2023/10 --- Essential People Skills for Line Managers --- CIPD, Cambridge, UK

    \item
    2022/11 --- Linux System Administration Essentials (LFS207) --- Linux Foundation, Online
    
    \item
    2021/08 --- ILM Level 3 Award in Leadership and Management Training --- LearningTree, Online
    
    \item
    2021/05 --- Comprehensive Verilog --- Duolos, Online
    
    \item
    2021/05 --- Embedded Systems Hardware and Software Design --- Duolos, Online
    
    \item
    2018/08 --- Data Science and Machine Learning Bootcamp --- Pivigo, London, UK
    
    \item
    2017/06 --- Neural Networks and Deep Learning --- Coursera, Online
    
    \item
    2015/09 --- Computational Condensed Matter Physics Bootcamp --- ICTP, Triest, Italy
    
\end{itemize}

% ==============================================================================
\cvsection{About Me}
% ==============================================================================

\cvevent{}{Work Style}{}{}

I greatly enjoy working on complex products/features that require a team with multi-domain
expertise.
I enjoy taking a lead in breaking down the complexity, investigating the user requirements,
planing the work with the team members, and working collaboratively throughout to the delivery.
Moreover, I find the iterative approach the most efficient in the long run as it minimizes
risks of mis-prioritisation.
In my practice I find it essential to communicate with stakeholders from other functions, such as
product, delivery, IT security, marketing, to have multiple viewpoints.

% During work on a project I would like to spend my time roughly like that:

% % Adapted from @Jake's answer from http://tex.stackexchange.com/a/82729/226
% % \wheelchart{outer radius}{inner radius}{comma-separated list of value/text width/color/detail}
% \wheelchart{1.5cm}{0.5cm}{
%     45/10em/accent!10/{Sole development},
%     25/10em/accent!30/{Collaborative/Team work},
%     10/10em/accent!50/{Planning/Roadmapping},
%     10/10em/accent!70/{Meeting stakeholders},
%     10/10em/accent!90/{Lower priority work}
% }

Constantly find balance between sole and team development depending on the phase of
development and maturity of the team collaboration.

To keep in touch with the other teams I intend to contribute to shared codebases and
learn about their solution designs over a whiteboard.

\divider

\cvevent{}{Personal}{}{}

My main priority in life is spending time with the family and passing my hobbies to my son.

We have lots of plans on getting into camping trips with canoeing and climbing adventures.
In general bodyboarding and snorkelling are also on the top of the list, but it doesn't
happen as much as I wish, maybe if my new job had remote opportunities...

When the weather is not that great, I find time for getting a few woodworking
projects done and playing non-deterministic board games.

Ah, yeah I can be quite a geek sometimes and have fun from math puzzle solving such as
"Project Euler" and "Advent of Code", especially with new languages.

% ==============================================================================
\end{paracol}
% ==============================================================================

% ==============================================================================
\end{document}
% ==============================================================================
