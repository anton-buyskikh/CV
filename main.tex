%% Use the "normalphoto" option if you want a normal photo instead of cropped to a circle
% \documentclass[10pt,a4paper,normalphoto]{altacv}

\documentclass[10pt,a4paper,ragged2e,withhyper]{altacv}
%% AltaCV uses the fontawesome5 and packages.
%% See http://texdoc.net/pkg/fontawesome5 for full list of symbols.

\geometry{left=1.25cm,right=1.25cm,top=1.5cm,bottom=1.5cm,columnsep=1.2cm}

% Set columns of text in parallel
\usepackage{paracol}

% Change the font, depending on whether you're using pdflatex or xelatex/lualatex
\ifxetexorluatex
  % If using xelatex or lualatex:
  \setmainfont{Roboto Slab}
  \setsansfont{Lato}
  \renewcommand{\familydefault}{\sfdefault}
\else
  % If using pdflatex:
  \usepackage[rm]{roboto}
  \usepackage[defaultsans]{lato}
  % \usepackage{sourcesanspro}
  \renewcommand{\familydefault}{\sfdefault}
\fi

% Change colours
\definecolor{SlateGrey}{HTML}{2E2E2E}
\definecolor{LightGrey}{HTML}{666666}
\definecolor{DarkPastelRed}{HTML}{450808}
\definecolor{PastelRed}{HTML}{8F0D0D}
\definecolor{DarkPastelGreen}{HTML}{450808}
\definecolor{PastelGreen}{HTML}{8F0D0D}
\definecolor{GoldenEarth}{HTML}{E7D192}

\colorlet{name}{black}
\colorlet{tagline}{PastelRed}
\colorlet{heading}{DarkPastelRed}
\colorlet{headingrule}{GoldenEarth}
\colorlet{subheading}{PastelRed}
\colorlet{accent}{PastelRed}
\colorlet{emphasis}{SlateGrey}
\colorlet{body}{LightGrey}

% Change fonts
\renewcommand{\namefont}{\Huge\rmfamily\bfseries}
\renewcommand{\personalinfofont}{\footnotesize}
\renewcommand{\cvsectionfont}{\LARGE\rmfamily\bfseries}
\renewcommand{\cvsubsectionfont}{\large\bfseries}

% Change bullets for itemize and rating marker for \cvskill
\renewcommand{\itemmarker}{{\small\textbullet}}
\renewcommand{\ratingmarker}{\faCircle}

% Can make fonts of itemize environments smaller
% \AtBeginEnvironment{itemize}{\small}

\begin{document}
\name{Anton Buyskikh}
\tagline{Senior Quantum Software Engineer}
\photoR{2.8cm}{figs/photo}

\personalinfo{%
    \email{anton.buyskikh@gmail.com}
    \phone{+44 7510 850009}
    \location{Cambridge, UK}
    \linkedin{anton-buyskikh}
    \github{anton-buyskikh}
    \orcid{0000-0003-4542-7086}
    \printinfo{\faFlag}{Citizenship: Russia \& UK}[]
    % \homepage{www.homepage.com}
    % \twitter{@twitterhandle}
    
    %% You can add your own arbitrary detail with
    %% \printinfo{symbol}{detail}[optional hyperlink prefix]
    % \printinfo{\faFlag}{Russian-British Nationality}[]
    %% Or you can declare your own field with
    %% \NewInfoFiled{fieldname}{symbol}[optional hyperlink prefix] and use it:
    % \NewInfoField{gitlab}{\faGitlab}[https://gitlab.com/]
    % \gitlab{your_id}
    %%
    %% For services and platforms like Mastodon where there isn't a
    %% straightforward relation between the user ID/nickname and the hyperlink,
    %% you can use \printinfo directly e.g.
    % \printinfo{\faMastodon}{@username@instace}[https://instance.url/@username]
    %% But if you absolutely want to create new dedicated info fields for
    %% such platforms, then use \NewInfoField* with a star:
    % \NewInfoField*{mastodon}{\faMastodon}
    %% then you can use \mastodon, with TWO arguments where the 2nd argument is
    %% the full hyperlink.
    % \mastodon{@username@instance}{https://instance.url/@username}
}

\makecvheader

Research Software Engineer with a wide experience of solution development for customers
seeking for new opportunities of bringing more value.
It's in my style to guide the solution from the early days of working with the customer
understanding their needs, striking the balance between high quality and cost constraints.
I currently develop the software stack for the near term quantum computers and have designed
and delivered applications and libraries used for internal and external products.
Being naturally good with communication and with attention to detail I like working on
with multifunctional teams, bridging hardware and software stack.
My previous experiences also include researching as a data scientist.

\columnratio{0.7}

\begin{paracol}{2}

% ==============================================================================
\cvsection{Relevant Experiences}
% ==============================================================================

\cvevent{Senior Quantum Software Engineer}{Riverlane}{2021/07 -- Present}{Cambridge, UK}
\cvevent{Quantum Software Engineer}{Riverlane}{2019/09 -- 2021/07}{Cambridge, UK}

\begin{itemize}

    \item
    Product owner of Benchkit for NQCC: working with stakeholders on
    specs, organized and lead the technical team, delivered on time,
    trained customers, and maintained the application.
    {\bf [Python, FastAPI, Django, MySQL, PHP, Docker]}
    
    \item
    Lead the technical side of the Calibrate product discovery with external
    customers.
    Developed multiple prototypes and convinced the stakeholders to integrate
    with the main stream of products.
    {\bf [Python, Airflow, PostgreSQL, Docker, Celery, Xilinx ZCU216 FPGA]}
    
    \item
    Lead development of the internal hardware modelling tool that allowed us
    to estimate performance of the hardware integration ahead of RTL implementation.
    {\bf [C++, SystemC, Python]}

    \item
    Low level integration of the software stack responsible for quantum error
    correction (QEC) cycles.
    {\bf [C++, gRPC, Python]}

    \item
    Developed the compliation toolchain for lowering abstraction of physical
    qubits to the pulse level of the target control system.
    {\bf [Python]}

    \item
    Product owner of the product life cycle: from planning
    and writing the spec together with external stakeholders to the delivery,
    providing external training, and working on follow up feedback.

    \item
    Working together with the product and delivery teams to facilitate the
    best solution to the customer.

    \item
    Lead by example, follow best practices, and proactively offer support to our
    junior developers, mentoring and line managing.

\end{itemize}

\divider

\cvevent{Data Scientist}{CitySprint}{2018/08 -- 2018/09}{London, UK}

Five weeks of intensive, project-based training turning exceptional analytical PhDs into Data Scientists.
In addition a "mini-MBA" program, data science fellows work on a commercial data science problem with a company.
Fellows are supported by business mentors, e.g. CTOs or Heads of Engineering/Analytics;
and technical mentors, experienced data scientists and technologists from industry,
and work in teams to develop high-impact, cutting-edge, sustainable, and scalable data science solutions.

\begin{itemize}
    \item Our team has developed a customer lifetime value (CLV) model and investigated the customers churn.
    \item Developed a predictive model providing a marketing strategy tool with potential savings
    for the company exceeding £1M per year.
\end{itemize}

\divider

\cvevent{Research Associate}{University of Strathclyde}{2017/08 -- 2019/09}{Glasgow, UK}

Numerical investigation of quantum many-body systems of cold atomic gases in- and out-of-equilibrium.
These systems have a great potential in quantum computing and quantum simulations that yet to be exploited.
I am involved in development and application of a Matlab library for one dimensional tensor network
algorithms – a state-of-art method allowing our team to trace the dynamics of the quantum system exactly.
I am actively investigating machine learning algorithms with a potential in the analysis of these systems.

\begin{itemize}
    \item Key achievements in my role.
    \item HPC with C++, Matlab, Python.
    \item Cluster management.
    \item Cosupervising and training.
    \item R\&D.
\end{itemize}

% ==============================================================================
\cvsection{Education}
% ==============================================================================

\cvevent{PhD in Physics}{University of Strathclyde}{2014 -- 2017}{Glasgow, UK}
Thesis: Dynamics of Quantum Many-Body Systems with Long-Range Interactions

\divider

\cvevent{MSc in Physics}{University of Pittsburgh}{2012 -- 2014}{Pittsburgh, PA, USA}

\divider

\cvevent{BSc\,\&\,MSc in Theoretical Physics}{St. Petersburg Polytechnic University}{2006 -- 2012}{St. Petersburg, Russia}
Thesis: THz radiation generation in AlGaAs semiconductor nanowires excited by femtosecond pulses.

% ==============================================================================
\cvsection{Other Experiences}
% ==============================================================================

\cvevent{Tutor}{University of Strathclyde}{2018/09 -- 2019/09}{Glasgow, UK}
Topics: Python and Matlab programming for undergraduate students.

\divider

\cvevent{Teacher Assistant}{University of Pittsburgh}{2012/08 -- 2014/06}{Pittsburgh, PA, USA}
Classes: Classical Mechanics, Quantum Mechanics

\divider

\cvevent{Lab Assistant}{Ioffe Institute}{2009/07 -- 2012/07}{St. Petersburg, Russia}
% Optical table stuff.

% ==============================================================================
\cvsection{Volunteering}
% ==============================================================================

\cvevent{Treasurer}{SCOPE}{2017/11 -- 2019/01}{Glasgow, UK}
\cvevent{Officer for Social Events}{SCOPE}{2015/10 -- 2017/11}{Glasgow, UK}

Student Community for Optics, Physics and Engineering (SCOPE) is dedicated
to enrich and enhance the postgraduate students experience by organising outreach and
educational activities, jobs fairs, company and school visits, journal clubs,
and social events.

\switchcolumn

% ==============================================================================
\cvsection{Programming Languages}
% ==============================================================================

\cvtag{Python}
\cvtag{C++}
\cvtag{Bash}
\cvtag{PostgreSQL}
\cvtag{Rust}
\cvtag{Matlab}
\cvtag{C\#}
\cvtag{SystemVerilog}
\cvtag{Fortran}
\cvtag{R}
\cvtag{Julia}
\cvtag{HTML}
\cvtag{JavaScript}
\cvtag{PHP}

% ==============================================================================
\cvsection{Software/Tools}
% ==============================================================================

\cvtag{Docker}
\cvtag{Airflow}
\cvtag{LLVM}
\cvtag{SystemC}
\cvtag{Azure}
\cvtag{Git}
\cvtag{gRPC}
\cvtag{MPI}
\cvtag{OpenMP}
\cvtag{Unity}
\cvtag{Django}
\cvtag{GitHub Actions}

% ==============================================================================
\cvsection{Other Skills}
% ==============================================================================

\cvtag{Agile development}
\cvtag{Embedded Systems}
\cvtag{FPGA}
\cvtag{Statistical Analysis}
\cvtag{Machine Learning}
\cvtag{Hardware Modelling}
\cvtag{SoC Design}
\cvtag{Compilation}
\cvtag{Linux}
\cvtag{CI/CD}

% ==============================================================================
\cvsection{Hobbies}
% ==============================================================================

\cvtag{Climbing}
\cvtag{Camping}
\cvtag{Woodworking}
\cvtag{Snowboarding}
\cvtag{Surfing}

\end{paracol}

\end{document}

% ==============================================================================
% END OF CV
% The following sections are good to have written down for interviews
% ==============================================================================

% ==============================================================================
\cvsection{Most Proud of}
% ==============================================================================

\cvachievement{\faTrophy}{End-to-end app development}
{Full life cycle of the application development, user interviews, requirements, specs, planning,
prototyping, demoing, building, delivering, feedback, bug fixing, training, and maintenance.}

\divider

\cvachievement{\faHeartbeat}{Product ownership}
{Developed and delivered to customer's hands two applications.}
% NQCC, Calibration

% ==============================================================================
\cvsection{Workload balance}
% ==============================================================================

% Adapted from @Jake's answer from http://tex.stackexchange.com/a/82729/226
% \wheelchart{outer radius}{inner radius}{comma-separated list of value/text width/color/detail}
\wheelchart{1.5cm}{0.5cm}{%
 4/8em/accent!30/{Collaborative work},
 3/8em/accent!40/Sole development,
 1/8em/accent!60/Planning,
 1/10em/accent/{Meeting\\stakeholders},
 1/6em/accent!20/Team building
}
