%% Use the "normalphoto" option if you want a normal photo instead of cropped to a circle
% \documentclass[10pt,a4paper,normalphoto]{altacv}

\documentclass[10pt,a4paper,ragged2e,withhyper]{altacv}
%% AltaCV uses the fontawesome5 and packages.
%% See http://texdoc.net/pkg/fontawesome5 for full list of symbols.

% Change the page layout if you need to
\geometry{left=1.25cm,right=1.25cm,top=1.5cm,bottom=1.5cm,columnsep=1.2cm}

% The paracol package lets you typeset columns of text in parallel
\usepackage{paracol}


% Change the font if you want to, depending on whether
% you're using pdflatex or xelatex/lualatex
\ifxetexorluatex
  % If using xelatex or lualatex:
  \setmainfont{Roboto Slab}
  \setsansfont{Lato}
  \renewcommand{\familydefault}{\sfdefault}
\else
  % If using pdflatex:
  \usepackage[rm]{roboto}
  \usepackage[defaultsans]{lato}
  % \usepackage{sourcesanspro}
  \renewcommand{\familydefault}{\sfdefault}
\fi

% Change the colours if you want to
\definecolor{SlateGrey}{HTML}{2E2E2E}
\definecolor{LightGrey}{HTML}{666666}
\definecolor{DarkPastelRed}{HTML}{450808}
\definecolor{PastelRed}{HTML}{8F0D0D}
\definecolor{DarkPastelGreen}{HTML}{450808}
\definecolor{PastelGreen}{HTML}{8F0D0D}
\definecolor{GoldenEarth}{HTML}{E7D192}

\colorlet{name}{black}
\colorlet{tagline}{PastelRed}
\colorlet{heading}{DarkPastelRed}
\colorlet{headingrule}{GoldenEarth}
\colorlet{subheading}{PastelRed}
\colorlet{accent}{PastelRed}
\colorlet{emphasis}{SlateGrey}
\colorlet{body}{LightGrey}


% Change some fonts, if necessary
\renewcommand{\namefont}{\Huge\rmfamily\bfseries}
\renewcommand{\personalinfofont}{\footnotesize}
\renewcommand{\cvsectionfont}{\LARGE\rmfamily\bfseries}
\renewcommand{\cvsubsectionfont}{\large\bfseries}

% Change the bullets for itemize and rating marker
% for \cvskill if you want to
\renewcommand{\itemmarker}{{\small\textbullet}}
\renewcommand{\ratingmarker}{\faCircle}

%% Use (and optionally edit if necessary) this .tex if you
%% want to use an author-year reference style like APA(6)
%% for your publication list
% % When using APA6 if you need more author names to be listed
% because you're e.g. the 12th author, add apamaxprtauth=12
\usepackage[backend=biber,style=apa6,sorting=ydnt]{biblatex}
\defbibheading{pubtype}{\cvsubsection{#1}}
\renewcommand{\bibsetup}{\vspace*{-\baselineskip}}
\AtEveryBibitem{%
  \makebox[\bibhang][l]{\itemmarker}%
  \iffieldundef{doi}{}{\clearfield{url}}%
}
\setlength{\bibitemsep}{0.25\baselineskip}
\setlength{\bibhang}{1.25em}


%% Use (and optionally edit if necessary) this .tex if you
%% want an originally numerical reference style like IEEE
%% for your publication list
\usepackage[backend=biber,style=ieee,sorting=ydnt]{biblatex}
%% For removing numbering entirely when using a numeric style
\setlength{\bibhang}{1.25em}
\DeclareFieldFormat{labelnumberwidth}{\makebox[\bibhang][l]{\itemmarker}}
\setlength{\biblabelsep}{0pt}
\defbibheading{pubtype}{\cvsubsection{#1}}
\renewcommand{\bibsetup}{\vspace*{-\baselineskip}}
\AtEveryBibitem{%
  \iffieldundef{doi}{}{\clearfield{url}}%
}


%% sample.bib contains your publications
%\addbibresource{sample.bib}

\begin{document}
\name{Anton Buyskikh}
\tagline{Senior Quantum Software Engineer}
%% You can add multiple photos on the left or right
\photoR{2.8cm}{prof_pic}

\personalinfo{%
    \email{anton.buyskikh@gmail.com}
    \phone{+44 7510 850009}
    \location{Cambridge, UK}
    \linkedin{anton-buyskikh}
    \github{anton-buyskikh}
    
    % \orcid{0000-0000-0000-0000}
    % \google-scholar{anton-buyskikh}
    % \homepage{www.homepage.com}
    % \twitter{@twitterhandle}
    
    %% You can add your own arbitrary detail with
    %% \printinfo{symbol}{detail}[optional hyperlink prefix]
    \printinfo{\faFlag}{Russian-British Nationality}[]
    %% Or you can declare your own field with
    %% \NewInfoFiled{fieldname}{symbol}[optional hyperlink prefix] and use it:
    % \NewInfoField{gitlab}{\faGitlab}[https://gitlab.com/]
    % \gitlab{your_id}
    %%
    %% For services and platforms like Mastodon where there isn't a
    %% straightforward relation between the user ID/nickname and the hyperlink,
    %% you can use \printinfo directly e.g.
    % \printinfo{\faMastodon}{@username@instace}[https://instance.url/@username]
    %% But if you absolutely want to create new dedicated info fields for
    %% such platforms, then use \NewInfoField* with a star:
    % \NewInfoField*{mastodon}{\faMastodon}
    %% then you can use \mastodon, with TWO arguments where the 2nd argument is
    %% the full hyperlink.
    % \mastodon{@username@instance}{https://instance.url/@username}
}

\makecvheader
%% Depending on your tastes, you may want to make fonts of itemize environments slightly smaller
% \AtBeginEnvironment{itemize}{\small}

\columnratio{0.65}

\begin{paracol}{2}

\cvsection{Experiences}

\cvevent{Senior Quantum Software Engineer}{}{July 2021 -- Present}{Cambridge, UK}
\cvevent{Quantum Software Engineer}{Riverlane}{Sep 2019 -- July 2021}{Cambridge, UK}

\begin{itemize}
    \item Product owner responsibilities (Deltaflow, Benchkit, Calibration).
    \item Taking ownership for the full development and deployment cycle: from planning
    and writing the spec together with external stakeholders to the delivery,
    providing external training, and working on follow up feedback.
    \item Driver for good software practices across the company, organiser of one of
    the community of practices.
    \item Working together with the product and delivery teams to facilitate the
    best solution to the customer.
    \item Mentoring and line managing junior colleagues. Lead by example, follow best
    practices and proactively offer support to our junior developers.
\end{itemize}

\divider

\cvevent{Data Scientist}{CitySprint}{2018/08 -- 2018/09}{London, UK}

Five weeks of intensive, project-based training turning exceptional analytical PhDs into Data Scientists.
In addition a "mini-MBA" program, data science fellows work on a commercial data science problem with a company.
Fellows are supported by business mentors, e.g. CTOs or Heads of Engineering/Analytics;
and technical mentors, experienced data scientists and technologists from industry,
and work in teams to develop high-impact, cutting-edge, sustainable, and scalable data science solutions.

\begin{itemize}

    \item Our team has developed a customer lifetime value (CLV) model and investigated the customers churn.

    \item Developed a predictive model providing a marketing strategy tool with potential savings
    for the company exceeding £1M per year.

\end{itemize}

\cvevent{Research Associate}{University of Strathclyde}{2017/08 -- 2019/09}{Glasgow, UK}


Numerical investigation of quantum many-body systems of cold atomic gases in- and out-of-equilibrium.
These systems have a great potential in quantum computing and quantum simulations that yet to be exploited.
I am involved in development and application of a Matlab library for one dimensional tensor network
algorithms – a state-of-art method allowing our team to trace the dynamics of the quantum system exactly.
I am actively investigating machine learning algorithms with a potential in the analysis of these systems.

\begin{itemize}
    \item Key achievements in my role.
    \item HPC with C++, Matlab, Python.
    \item Cluster management.
    \item Cosupervising and training.
    \item R\&D.
\end{itemize}

\cvsection{Education}

\cvevent{Ph.D.\ in Your Discipline}{Your University}{Sept 2002 -- June 2006}{}
Thesis title: Long range Many-Body Physics

\divider

\cvevent{M.Sc.\ in Theoretical physics}{Pitt}{Sept 2012 -- June 2014}{}

\divider

\cvevent{M.Sc.\ in Condensed matter physics}{AFTU}{Sept 2010 -- June 2012}{}

\divider

\cvevent{B.Sc.\ in Theoretical Physics}{Polytech}{Sept 2006 -- June 2010}{}

\cvsection{Other Experiences}

\cvevent{Teacher Assistant}{University of Pittsburgh}{Aug 2012 -- June 2014}{Pittsburgh, USA}

%\cvsection{Workload balance}

% Adapted from @Jake's answer from http://tex.stackexchange.com/a/82729/226
% \wheelchart{outer radius}{inner radius}{comma-separated list of value/text width/color/detail}
\wheelchart{1.5cm}{0.5cm}{%
 4/8em/accent!30/{Collaborative work},
 3/8em/accent!40/Sole development,
 1/8em/accent!60/Planning,
 1/10em/accent/{Meeting\\stakeholders},
 1/6em/accent!20/Team building
}

\switchcolumn

\cvsection{Top Languages}

\cvskill{Python}{5}
\cvskill{C++}{4}
\cvskill{Rust}{3.5}

\cvsection{Software/Tools}

\cvskill{Docker}{5}
\cvskill{Azure}{4}
\cvskill{Git}{5}
\cvskill{CI/CD}{5}

\cvsection{Other Skills}

\cvtag{C++}
\cvtag{Embedded Systems}
\cvtag{Statistical Analysis}

\cvsection{Strengths}

\cvtag{Communication skills}
\cvtag{Proactive}
\cvtag{Hard-working}
\cvtag{Eye for detail}
\cvtag{Motivator \& Leader}

\cvsection{Other skills}

\cvtag{Communication skills}
\cvtag{Proactive}
\cvtag{Hard-working}
\cvtag{Eye for detail}\\
\cvtag{Motivator \& Leader}

\cvsection{Hobbies}

\cvtag{Communication skills}
\cvtag{Proactive}
\cvtag{Hard-working}
\cvtag{Eye for detail}\\
\cvtag{Motivator \& Leader}

\end{paracol}

\end{document}

% \cvsection{Projects}
% \cvevent{Project 1}{Funding agency/institution}{}{}
% \begin{itemize}
% \item Details
% \end{itemize}
% \divider
% \cvevent{Project 2}{Funding agency/institution}{Project duration}{}
% A short abstract would also work.
% \medskip

% \cvsection{Most Proud of}
% \cvachievement{\faTrophy}{Fantastic Achievement}{and some details about it}
% \divider
% \cvachievement{\faHeartbeat}{Another achievement}{more details about it of course}
% \divider
% \cvachievement{\faHeartbeat}{Another achievement}{more details about it of course}
